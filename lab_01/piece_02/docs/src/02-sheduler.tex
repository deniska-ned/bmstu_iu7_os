\chapter{Пересчет динамических приоритетов}

    \section{Unix/Linux}
    
        \subsection{Приоритеты процессов}
        
            Приоритет  процесса  задается  любым  целым  числом,  лежащим  в  диапазоне  от  0  до  127.  Чем  меньше  такое  число,  тем  выше  приоритет.  Приоритеты  от  0  до  49  зарезервированы  для  ядра,  следовательно, прикладные  процессы  могут  обладать  приоритетом  в  диапазоне  50-127.  Структура  proc  содержит  следующие  поля,  относящиеся  к  приоритетам: 
            
            \begin{itemize}
                \item p\_pri -- Текущий  приоритет  планирования
                \item р\_usrpri -- Приоритет  режима  задачи 
                \item p\_cpu -- Результат  последнего  измерения  использования  процессора 
                \item p\_nice -- Фактор  «любезности»,  устанавливаемый  пользователем 
            \end{itemize}
        
            Планировщик  использует  p\_pri  для  принятия  решения  о  том,  какой  процесс  направить  на  выполнение.  Когда  процесс  находится  в  режиме  задачи,  значение  его  p\_pri  идентично  p\_usrpri.  Когда  процесс  просыпается  после  блокирования  в  системном  вызове,  его  приоритет  будет  временно  повышен  для  того,  чтобы  дать  ему  предпочтение  для  выполнения  в  режиме  ядра.  Следовательно,  планировщик  использует  p\_usrpri  для  хранения  приоритета,  который  будет  назначен  процессу  при  возврате  в  режим  задачи,  а  p\_pri —  для  хранения  временного  приоритета  для  выполнения  в  режиме  ядра.
            
            Приоритет  в  режиме  задачи  зависит  от  двух  факторов:  «любезности»  (nice)  и  последней  измеренной  величины  использования  процессора.  Степень  любезности  (nice)  является  числом  в  диапазоне  от  0  до  39  со  значением  20  по  умолчанию.  Увеличение  значения  приводит  к  уменьшению  приоритета.  Фоновым  процессам  автоматически  задаются  более  высокие  значения  степени  благоприятствия.  Уменьшить  эту  величину  для  какого-либо  процесса  может  только  суперпользователь,  поскольку  при  этом  поднимется  его  приоритет.
            
            Поле  p\_cpu  структуры  proc  содержит  величину  результата  последнего  сделанного  измерения  использования  процессора  процессом.  При  создании  процесса  значение  этого  поля  инициализируется  нулем.  На  каждом  тике  обработчик  таймера  увеличивает  p\_cpu  на  единицу  для  текущего  процесса  до  максимального  значения,  равного  127.
            
            В современных UNIX ядро является полностью вытесняемым. Например, \(Solaris 2.x\), \(Linux\).

        \subsection{Перерасчет приоритетов}
            Каждую  секунду  ядро  системы  вызывает  процедуру  schedcpu()  (запускаемую  через  отложенный  вызов),  которая  уменьшает  значение  p\_cpu  каждого  процесса  исходя  из  фактора  «полураспада»  (delay factor).  В  системе  SVR3  используется  фиксированное  значение  этого  фактора,  равное \(\frac{1}{2}\).  
            
            Процедура  schedcpu()  пересчитывает  приоритеты  для  режима  задачи  всех  процессов  по  формуле
            
            \[
            p\_usrpri = PUSER + \frac{p\_cpu}{4} + 2 \cdot p\_nice
            \]
            
            где  PUSER  —  базовый  приоритет  в  режиме  задачи,  равный  50.  
            
            Процедура  schedcpu()  пересчитывает  приоритет  каждого  процесса  каждую  секунду.  Так  как  приоритет  процесса,  находящегося  в  очереди  на  выполнение,  не  может  быть  изменен,  процедура  schedcpu()  извлекает  процесс  из  очереди,  меняет  его  приоритет  и  помещает  его  обратно.  Обработчик  прерываний  таймера  пересчитывает  приоритет  текущего  процесса  через  каждые  четыре  тика. 

            Ядро системы связывает приоритет сна с событием или ожидаемым ресурсом, из-за которого процесс может блокироваться. Когда процесс просыпается после блокирования в системном вызове, ядро устанавливает в поле p\_pri приоритет сна – значение приоритета из диапазона от 0 до 49, зависящее от события или ресурса по которому произошла блокировка. На рисунке 2.1 показано событие и связанное с ним значение приоритета сна в системе 4.3BSD.
            
            \imgw{pvd.png}{htb!}{\textwidth}{Системы приоритета сна 4.3BSD}
            
        \subsection{Переключение контекста}
        
            Существуют  три  ситуации,  при  которых  возникает  переключение  контекста. 
            
            \begin{itemize}
                \item Если  текущий  процесс  блокируется  в  ожидании  ресурса  или  завершает  работу.  При  этом  происходит  свободное  переключение  контекста. 
                \item Если  в  результате,  полученном  процедурой  пересчета  приоритетов,  оказалось,  что  другой  процесс  обладает  более  высоким  приоритетом  по  сравнению  с  текущим. 
                \item Если  текущий  процесс  или  обработчик  прерываний  разбудил  более  приоритетный  процесс.
            \end{itemize}

    \section{Windows}
    
        В Windows при создании процесса, ему назначается базовый приоритет. Относительно базового приоритета процесса потоку назначается относительный приоритет.
        
        Планирование осуществляется на основании приоритетов потоков, готовых к выполнению. Поток с более низким приоритетом вытесняется планировщиком, когда поток с более высоким приоритетом становится готовым к выполнению. По истечению кванта времени текущего потока, ресурс передается первому — самому приоритетному — потоку в очереди готовых на выполнение.
        
        Раз в секунду диспетчер настройки баланса сканирует очередь готовых потоков. Если обнаружены потоки, ожидающие выполнения более 4 секунд, диспетчер настройки баланса повышает их приоритет до 15. Как только квант истекает, приоритет потока снижается до базового приоритета. Если поток не был завершен за квант времени или был вытеснен потоком с более высоким приоритетом, то после снижения приоритета поток возвращается в очередь готовых потоков.
        
        Чтобы минимизировать расход процессорного времени, диспетчер настройки баланса сканирует лишь 16 готовых потоков. Кроме того, диспетчер повышает приоритет не более чем у 10 потоков за один проход: обнаружив 10 потоков, приоритет которых следует повысить, он прекращает сканирование. При следующем проходе сканирование возобновляется с того места, где оно было прервано в прошлый раз. Наличие 10 потоков, приоритет которых следует повысить, говорит о необычно высокой загруженности системы
        
        В Windows используется 32 уровня приоритета: целое число от 0 до 31, где 31 — наивысший приоритет, из них:
        
        \begin{itemize}
            \item от 16 до 31 — уровни реального времени
            \item от 0 до 15 — динамические уровни, уровень 0 зарезервирован для потока обнуления страниц
        \end{itemize}
        
        Уровни приоритета потоков назначаются исходя из двух разных позиций: одной от Windows API и другой от ядра Windows. Сначала Windows API систематизирует процессы по классу приоритета, который им присваивается при создании: Реального времени — Real-time (4), Высокий — High (3), Выше обычного — Above Normal (7), Обычный — Normal (2), Ниже обычного — Below Normal (5) и Простоя — Idle (1).
        
        Затем назначается относительный приоритет отдельных потоков внутри этих процессов. Здесь номера представляют изменение приоритета, применяющееся к базовому приоритету процесса: Критичный по времени — Time-critical (15), Наивысший — Highest (2), Выше обычного — Above-normal (1), Обычный — Normal (0), Ниже обычного — Below-normal (–1), Самый низший — Lowest (–2) и Простоя — Idle (–15).

        \imgw{win_pri_levels.png}{htb!}{\textwidth}{Уровни приоритета потоков}
        
        Исходный базовый приоритет потока наследуется от базового приоритета процесса. Процесс по умолчанию наследует свой базовый приоритет у того процесса, который его создал. Соответствие между приоритетами Windows API и ядра системы приведено на таблице \ref{tbl:priority}.
        
        \begin{table}[h]
            \caption{Соответствие между приоритетами Windows API и ядра Windows}
            \begin{center}
                \begin{tabular}{|l|p{45pt}|p{45pt}|p{45pt}|p{45pt}|p{45pt}|p{45pt}|}
                    \hline
                    {} & real-time & high & above normal & normal & below normal & idle\\
                    \hline
                    time critical & 31 & 15 & 15 & 15 & 15 & 15 \\
                    \hline
                    highest & 26 & 15 & 12 & 10 & 8 & 6 \\
                    \hline
                    above normal & 25 & 14 & 11 & 9 & 7 & 5 \\
                    \hline
                    normal & 24 & 13 & 10 & 8 & 6 & 4 \\
                    \hline
                    below normal & 23 & 12 & 9 & 7 & 5 & 3 \\
                    \hline
                    lowest & 22 & 11 & 8 & 6 & 4 & 2 \\
                    \hline
                    idle & 16 & 1 & 1 & 1 & 1 & 1 \\
                    \hline
                \end{tabular}
            \end{center}
            \label{tbl:priority}
        \end{table}
        
        
        Текущий приоритет потока в динамическом диапазоне — от 1 до 15 — может быть повышен планировщиком вследствие следующих причин:
        
        \begin{itemize}
            \item  повышение вследствие событие планировщика или диспетчера(сокращение задержек);
            \item повышение приоритета владельца блокировки;
            \item повышение приоритета после завершения ввода/вывода (сокращение задержек) (таблица \ref{tab:recom});
            \item повышение приоритета вследствие ввода из пользовательского интерфейса(сокращение задержек и времени отклика);
            \item повышение приоритета вследствие длительного ожидания ресурса исполняющей системы(предотвращение зависания);
            \item повышение вследствие ожидания объекта ядра;
            \item повышение приоритета в случае, когда готовый к выполнению поток не был запущен в течение длительного времени (предотвращение зависания и смены приоритетов);
            \item повышение приоритета проигрывания мультимедиа службой планировщика MMCSS
        \end{itemize}
        
        \begin{table}[h]
            \caption{Рекомендуемые значения повышения приоритета}
            \begin{center}
                \begin{tabular}{|p{80mm}|p{40mm}|}
                    \hline
                    Устройство & Повышение приоритета \\
                    \hline
                    Жесткий диск, привод компакт-дисков, параллельный порт, видео устройство & 1 \\
                    \hline
                    Сеть, почтовый порт, именованный канал, последовательный порт & 2 \\
                    \hline
                    Клавиатура, мышь & 6 \\
                    \hline
                    Звуковое устройство & 8 \\
                    \hline
                \end{tabular}
            \end{center}
            \label{tab:recom}
        \end{table}

        \subsection{MMCSS}
        
            Повышение приоритета проигрывания мультимедиа обычно управляется службой пользовательского режима, которая называется службой планировщика класса мультимедиа — MultiMedia Class Scheduler Service (MMCSS). MMCSS работает с вполне определенными задачами, включая следующие
    
            \begin{itemize}
                \item аудио;
                \item игры;
                \item построчное воспроизведение текущего изображения в видеобуффер;
                \item воспроизведение медиа-контента;
                \item задачи администратора многоэкранного режима.
            \end{itemize}
            
            В свою очередь, каждая из этих задач включает информацию о свойствах, отличающих их друг от друга. Одно из наиболее важных свойств для планирования потоков называется категорией планирования — Scheduling Category, которое является первичным фактором, определяющим приоритет потоков, зарегистрированных с MMCSS. На таблице \ref{tab:plan} показаны различные категории планирования.
            
            Механизм, положенный в основу MMCSS, повышает приоритет потоков внутри зарегистрированного процесса до уровня, соответствующего их категории планирования.
            
            Затем он снижает категорию этих потоков до Exhausted, чтобы другие, не относящиеся к мультимедийным приложениям потоки, также могли получить ресурс
            
            \begin{table}[h]
                \caption{Категории планирования.}
                \begin{center}
                    \begin{tabular}{|p{40mm}|p{30mm}|p{80mm}|}
                        \hline
                        Категория & Приоритет & Описание\\
                        \hline
                        High (Высокая) & 23-26 & Потоки профессионального аудио (Pro Audio), запущенные с приоритетом выше, чем у других потоков на системе, за исключением критических системных потоков \\
                        \hline
                        Medium (Средняя) & 16-22 & Потоки, являющиеся частью приложений первого плана, например Windows Media Player \\
                        \hline
                        Low (Низкая) & 8-15 & Все остальные потоки, не являющиеся частью предыдущих категорий \\
                        \hline
                        Exhausted (Исчерпавших потоков) & 1-7 & Потоки, исчерпавшие свою долю времени центрального процессора, выполнение которых продолжиться, только если не будут готовы к выполнению другие потоки с более высоким уровнем приоритета \\
                        \hline
                    \end{tabular}
                \end{center}
                \label{tab:plan}
            \end{table}

\chapter{Функции обработчика прерывания от системного таймера}

    \textbf{Тик} --- период времени между двумя последующими прерываниями таймера.
    
    \textbf{Главный тик} --- период времени между \(N\) последующими прерываниями таймера. Число \(N\) зависит от конкретного варианта системы.

    \textbf{Квант времени} --- временной интервал, в течение которого процесс может использовать процессор до вытеснения другим процессом.

    \section{Unix/Linux}
    
        Обработчик прерывания от системного таймера выполняет следующие функции
    
        \subsection{По тику}
        
            \begin{itemize}
                \item Ведет счёт тиков аппаратного таймера по необходимости
                \item Обновляет статистику использования процессора текущим процессом.  Происходит инкремент поля p\_cpu структуры proc текущего процесса. Данное поле при создании процесса инициализируется нулем, а увеличивается до максимального значения, равного 127.
                \item Инкремент часов и других таймеров системы. В SVR4 происходит инкремент переменной lbolt, хранящей количество тиков, отсчитанных с момента загрузки системы.
                \item Декремент кванта
            \end{itemize}
        
        \subsection{По главному тику}
        
            \begin{itemize}
                \item Пробуждает в нужные моменты системные процессы, такие как swapper и pagedaemon. Под пробуждением понимается регистрация отложенного вызова процедуры wakeup, которая перемещает дескрипторы процессов из списка “спящих” в очередь готовых к выполнению.
                \item Декрементирует счётчик времени, оставшегося времени до посылки одного из следующих сигналов:
                \begin{itemize}
                    \item \(SIGALRM\) – сигнал, посылаемый процессу по истечении времени, предварительно заданного функцией alarm();
                    \item \(SIGPROF\) – сигнал, посылаемый процессу по истечении времени заданного в таймере профилирования;
                    \item \(SIGVTALRM\) – сигнал, посылаемый процессу по истечении времени, заданного в виртуальном таймере.
                \end{itemize}
            \end{itemize}
        
        \subsection{По кванту}
            \begin{itemize}
                \item Посылает текущему процессу сигнал \( SIGXCPU \), если тот превысил выделенную ему квоту использования процессора
            \end{itemize}
        
    \section{Windows}
    
        Обработчик прерывания от системного таймера выполняет следующие функции
    
        \subsection{По тику}
        
            \begin{itemize}
                \item Инкремент счетчика системного времени
                \item Декремент счетчика времени отложенные задач
                \item Декремент кванта текущего потока на величину, равную количеству тактов процессора, произошедших за тик (если количество затраченных потоком тактов процессора достигает квантовой цели, запускается обработка истечения кванта)
                \item если активен механизм профилирования ядра, то инициализирует отложенный вызов обработчика ловушки профилирования ядра путём постановки объекта в очередь DPC(Deferred procedure call (отложенный вызов процедуры)): обработчик ловушки профилирования регистрирует адрес команды, выполнявшейся на момент прерывания.
            \end{itemize}
        
        \subsection{По главному тику}
        
            \begin{itemize}
                \item Освобождает объект <<событие>>, который ожидает диспетчер настройки баланса.
            \end{itemize}
        
        \subsection{По кванту}
    
            \begin{itemize}
                \item Инициализация диспечерезации потоков путем постановки соответствующего объекта DPC в очередь
            \end{itemize}
        